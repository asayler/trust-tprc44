\section{Modeling Trust}
\label{sec:model}

Generally, when users leverage modern computing devices and services,
they must trust third party manufacturers and service providers with
their data. The privacy of user data relies on this trust via has two
main factor: how much trust must a user place in third parties
(e.g. how much of their personal data do they grant the third party
access to), and in what manners can the third party violate this trust
(e.g. how can the third party abuse the access they have been
granted). A model of third party trust must therefore evaluate trust
across two axes: the \emph{degree} of trust users must place in third
parties, and the manner in which this trust might be
\emph{violated}. The ideal privacy and security enhancing trust model
for a given use case will minimize the degree of third party trust
while also minimizing the likelihood that such trust will be violated.

In terms of degree of trust, third parties can be trusted with the
following data-related capabilities:

\begin{packed_desc}
\item[Storage (S):] \hfill \\ Can a third party faithfully store
  private user data and make it available to the user upon request?
  Misuse of this capability may result in a loss of user data, but
  won't necessarily result in the exposure of user data.
\item[Access (R):] \hfill \\ Can a third party read and interpret the
  private user data they store? Misuse of this capability may result
  in the exposure of user data.
\item[Manipulation (W):] \hfill \\ Can a third party modify the
  private user data to which they have access? Misuse of this
  capability may result in the ability to manipulate a user
  (e.g. changing appointments on a user's calendar, etc).
\item[Meta-analysis (M):] \hfill \\ Can a third party gather user
  metadata related to any stored user data or user behavior
  interacting with this data? Misuse of this capability may result in
  the ability to infer private user data (e.g. who a user's friends
  are based on data sharing patterns).
\end{packed_desc}

Trust violation occurs when a third party exercises any of the above
capabilities without explicit user knowledge and permission. Put
another way, a trust violation occurs whenever a third party leverages
a capability with which they are entrusted in a manner in which the
user does not expect the capability to be leveraged.

There are several types of trust violations. Each is defined by the
manner in which the violation occurs and the motivations behind it:

\begin{packed_desc}
\item[Implicit (P):] \hfill \\ This class of trust violation occurs
  when a third party violates a user's trust in a manner approved by
  the third party. An example might be sharing user data with a
  business partner (e.g. an advertiser). Often these forms of
  violations aren't really ``violations'' in the sense that a user may
  have clicked through a Terms of Service agreement that granted
  implicit permission for such use, but if the third party is engaging
  in behavior that the user would not generally expect, an implicit
  trust violation has occurred.
\item[Compelled (C):] \hfill \\ This class of trust violation occurs
  when a third party is compelled by another actor to violate a user's
  trust. The most common example would be a third party being forced
  to turn over user data or records in response to a request from the
  government with jurisdiction over the party. Another example might
  include a company going bankrupt and being forced to sell its user
  data to another entity~\cite{solove2015}.
\item[Unintentional (U):] \hfill \\ This form of violation occurs when
  a third party unintentionally discloses or manipulates user data. An
  example would be a coding error that allows either the loss of or
  unfettered access to user data.
\item[Colluding (L):] \hfill \\ This class of violation occurs when
  multiple third parties collude to gain capabilities over user data
  beyond what the user intended each to have individually. (An example
  would be an SSP sharing the user's encryption keys with the feature
  provider storing the corresponding encrypted data.)
\end{packed_desc}
