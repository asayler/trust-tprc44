\section{Managing Trust}
\label{sec:mitigation}

The current trust situation inherent in using many cloud --
i.e. trusting many third parties with a wide range of capabilities and
only moderate disincentivizes to violating user trust -- is far from
ideal. This state places private user data and metadata at a high
degree of risk for unapproved exposure or manipulation. It is natural
to ask what solutions might aid in better controlling third party
trust arraignments, reducing this degree of risk involved when
leveraging third party services. While there are a myriad of potential
solutions in this space, ranging form technical to policy, I suggest a
few high level approaches to managing third party trust in this
section.

The trust model presented in \S~\ref{sec:model} discusses to
components of third party trust: the capabilities we entrust to third
parties and the manners in which this trust might be violated. Further
control of third party trust can be exerted to either or both of these
axis. By reducing the degree or trust -- i.e. limiting the number of
capabilities third parties are granted -- users can limit the amount
of harm a third party can inflict should this trust be violated. By
disincentivizing the various types of trust violations, a user can
decrees the likelihood that a third party violated there trust at
all. I'll focus on strategies for each of these goals below.

\subsection{Limiting Capabilities}



Distributed Trust
Tor

Control Degree -> SSaaS, Etc
Disincentivize Violation -> Liability, Policy, Markets
