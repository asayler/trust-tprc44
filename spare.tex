

%%  Due to the CA
%% trust requirements that exists in such services' sharing
%% implementations, however, it is possible that such a service could be
%% compelled to mount a MitM attack on one of their users in order to
%% provide such data to the compelling party (similar to government
%% efforts to compel Apple to create a flawed version of iOS that would
%% be vulnerable to brute force attacks~\cite{ars-cookvfbi}). Similarly,
%% such a service could be compelled to surrender their CA private key,
%% allowing the compelling party to take over the trusted CA role and
%% mount such a MitM attack themselves (similar to the how Lavabit was
%% compelled to turn over their private TLS keys to facilitate government
%% access to the data they controlled~\cite{levsion-lavabit}). The fact
%% that such capabilities have been exploited for the purpose of
%% committing compelled violations in the past raises the likelihood of
%% such violations when using services such as Tresorit in the
%% future. Unintentional, Insider, and Outsider violations are all
%% similar to the traditional use case: such violations are technically
%% possible, but the third party has a vested interest in avoiding such
%% violations for reputation-related reasons. Collusion violations aren't
%% really applicable since Tresorit is a single-party actor.


%% It is possible such cloud communication services are actually at an
%% increased risk of compelled violations relative to other cloud
%% services due to the existence of laws such as the Communications
%% Assistance for Law Enforcement Act (CALEA)~\cite{calea-usc,
%% calea-fcc} specifically designed to aid the government in obtaining
%% a user's communications.

%% Google has taken steps to ensure
%% communication between all of its data centers and between mail
%% servers are encrypted in transit, helping to minimize outsider
%% violations~\cite{gmail-blog-encryption}. Similarly, it has recently
%% started providing end users with information as to whether or not
%% encrypted email transmission was possible and whether or not an
%% email's sender has been authenticated~\cite{gmail-blog-indicators}.
%% In both cases however, third parties, not end users, remain in full
%% control of all the necessary cryptographic keys, limiting this
%% protection to the mitigation of outsider violations, and doing
%% little to mitigate Implicit, Compelled, Unintentional, or Insider
%% violations.

%% In terms of likelihood of trust violations, LastPass has a very
%% similar profile to a service like Tresorit. LastPass is a ``freemium''
%% service that generates its income off its ability to faithfully store
%% and protect user passwords, encouraging users to pay for higher level
%% service tiers. This disincentivizes implicit trust
%% violations. LastPass also has a strong incentive to minimize (at least
%% public) compelled violations, although it is still subject to such
%% violations. Unintentional, insider, and outsider violations are
%% similarly disincentivized, although they have occurred
%% before~\cite{lastpass-blog-breach}.
