%% Unintentional (\emph{U}), Insider (\emph{I}), or Outsider (\emph{O})
%% violations are all possibilities when using Dropbox. On the \emph{U}
%% front, Dropbox had an incident in 2011 that allowed anyone to log into
%% the service using any password for a five hour
%% period~\cite{dropbox-authbug}. So far, Dropbox appears to have
%% avoided any \emph{I}-type violations, but it has been the target of
%% various \emph{O}-type attempted violations, primarily built around
%% advisories who obtain common user
%% passwords~\cite{dropbox-passwords}. Needless to say, while Dropbox
%% works to avoid these kinds of violations, they are certainly still
%% possible, have occurred in the past, and may well occur in the future.


%% In terms of trust violations, such ``secure'' cloud storage services
%% are being paid to store user data, disincentivizing implicit
%% violations. Compelled violations are possible, although as services
%% that market themselves as ``privacy'' products, compelled violations
%% (assuming they are public) are also disincentivized. Due to the CA
%% trust requirements that exists in such services' sharing
%% implementations, however, it is possible that such a service could be
%% compelled to mount a MitM attack on one of their users in order to
%% provide such data to the compelling party (similar to government
%% efforts to compel Apple to create a flawed version of iOS that would
%% be vulnerable to brute force attacks~\cite{ars-cookvfbi}). Similarly,
%% such a service could be compelled to surrender their CA private key,
%% allowing the compelling party to take over the trusted CA role and
%% mount such a MitM attack themselves (similar to the how Lavabit was
%% compelled to turn over their private TLS keys to facilitate government
%% access to the data they controlled~\cite{levsion-lavabit}). The fact
%% that such capabilities have been exploited for the purpose of
%% committing compelled violations in the past raises the likelihood of
%% such violations when using services such as Tresorit in the
%% future. Unintentional, Insider, and Outsider violations are all
%% similar to the traditional use case: such violations are technically
%% possible, but the third party has a vested interest in avoiding such
%% violations for reputation-related reasons. Collusion violations aren't
%% really applicable since Tresorit is a single-party actor.

%% Furthermore, since services such as Tresorit are not open source or
%% widely audited by independent parties, the user must trust that the
%% code they are running to access these services is faithfully
%% implementing the security guarantees the services claim to
%% support. It is possible that such a service could ship a flawed
%% version of their code due to a wide range of trust violations
%% (e.g. they could be compelled to ship such code, or it could be an
%% honest unintentional mistake).

%% In terms of likelihood of trust violations, Facebook differs a bit
%% from cloud-based system like Dropbox in that it is an ad-supported
%% system. This increases the likelihood of implicit trust failures due
%% to Facebook's business model being based around the sale of user
%% data. Indeed, and as mentioned in Chapter~\ref{chap:challenges},
%% Facebook has a record of implicit trust violations ranging from the
%% Emotional Contagion Study~\cite{goel2014} to its use of users' photos
%% in the ads it serves~\cite{mashable-socialads}. Beyond implicit trust
%% violations, Facebook's trust violation profile is again similar to
%% other traditional cloud services such as Dropbox. Like
%% Dropbox~\cite{dropbox-transparency}, compelled violations are known to
%% occur~\cite{facebook-transparency}. Unintentional, insider, and
%% outsider violations are all possible, but Facebook has an active
%% interest in avoiding them. Collusion violations aren't really
%% applicable since Facebook is a single-party actor.


%% Like Facebook, both services are ad-supported, increasing the
%% likelihood of implicit trust violations (although unlike Facebook,
%% Google seems to have a better track record of not committing such
%% violations). Similarly, both services are known to be subject to
%% compelled violations~\cite{google-transparency},\footnote{It is
%%   possible such cloud communication services are actually at an
%%   increased risk of compelled violations relative to other cloud
%%   services due to the existence of laws such as the Communications
%%   Assistance for Law Enforcement Act (CALEA)~\cite{calea-usc,
%%     calea-fcc} specifically designed to aid the government in
%%   obtaining a user's communications.}and are disincentivized, although
%% by no means immune, from committing unintentional, insider, and
%% outsider violations.\footnote{Google has taken steps to ensure
%%   communication between all of its data centers and between mail
%%   servers are encrypted in transit, helping to minimize outsider
%%   violations~\cite{gmail-blog-encryption}. Similarly, it has recently
%%   started providing end users with information as to whether or not
%%   encrypted email transmission was possible and whether or not an
%%   email's sender has been authenticated~\cite{gmail-blog-indicators}.
%%   In both cases however, third parties, not end users, remain in full
%%   control of all the necessary cryptographic keys, limiting this
%%   protection to the mitigation of outsider violations, and doing
%%   little to mitigate Implicit, Compelled, Unintentional, or Insider
%%   violations.}Thus, Google is able to analyze, alter, and expose any
%% data that travels across its network using either service -- either of
%% its own volition or due to a compelled order or unauthorized data
%% breach.


%% Since the OpenPGP protocol and associated implementations are not
%% controlled by any single third party, there is no straightforward
%% analysis of its likelihood of PGP-related trust violations. In
%% general, however, PGP implementations are controlled by parties with a
%% vested interest in maintaining user security and privacy. Such parties
%% are strongly disincentivized from committing implicit or compelled
%% violations. Most OpenPGP implementations are also open source. This
%% helps to mitigate\footnote{Although it does not necessarily prevent
%%   such violations, e.g. as in Heartbleed~\cite{heartbleed} or a Ken
%%   Thompson style attack~\cite{thompson1984}.}unintentional, insider,
%% and outsider violations by maximizing the number of eyes on the code
%% and reducing the likelihood of mistakes or intentionally coded
%% vulnerabilities and back doors.


%% Like OpenPGP, TextSecure is a protocol implemented by several
%% different apps. Such application providers (e.g. Open Whisper Systems)
%% specifically market their products as being a secure alternative to
%% more traditional chat systems, and are thus strongly dissuaded from
%% committing any kind of implicit or compelled trust
%% violation. Furthermore, most TextSecure implementations are open
%% source, which mitigates the likelihood of
%% unintentional,\footnote{Researchers have discovered bugs in
%%   TextSecure, but these bugs were quickly
%%   patched~\cite{frosch2014}.}insider, or outsider
%% violations. TextSecure has also undergone external audits and reviews,
%% further decreasing the likelihood of an unintentional or insider
%% violation~\cite{frosch2014}. As in OpenPGP, even if a violation
%% occurred, the capability restrictions discussed above severely limit
%% what data the violation would expose. Therefore, TextSecure represents
%% a successful effort to reduce third party trust exposure and secure
%% user communications.



%% In terms of likelihood of trust violations, LastPass has a very
%% similar profile to a service like Tresorit. LastPass is a ``freemium''
%% service that generates its income off its ability to faithfully store
%% and protect user passwords, encouraging users to pay for higher level
%% service tiers. This disincentivizes implicit trust
%% violations. LastPass also has a strong incentive to minimize (at least
%% public) compelled violations, although it is still subject to such
%% violations. Unintentional, insider, and outsider violations are
%% similarly disincentivized, although they have occurred
%% before~\cite{lastpass-blog-breach}.

%% Like Tresorit, LastPass is neither open source\footnote{Although parts
%%   of LastPass are build using JavaScript, which by its very nature
%%   exposes the source code to users. Nonetheless, such code can be
%%   obfuscated to make it difficult to read, and users lack most of the
%%   ``Free Software'' rights~\cite{fsf-freedoms} traditionally
%%   associated with ``open source'' code.}nor subject to public code
%% audits. As such, the user is required to trust that LastPass has
%% faithfully coded its software not to expose additional capabilities
%% via hidden back doors. It is possible that LastPass could be compelled
%% to modify the LastPass client program to do something like send copies
%% of a user's master password back to LastPass where it could be used to
%% decrypt all of the user's data. It is also possible that the lack of
%% cryptographic authentication on user data would allow LastPass to
%% mount a range of attacks on a user - potentially revealing one or more
%% of the passwords they store with LastPass in the process.
