\section{Conclusion}
\label{sec:conclusion}

The pervasiveness of third parties across the modern cloud computing
landscape is undeniable. What this pervasiveness means for the privacy
and security of users and their data is an area of active research. In
this paper, I presented a bi-axial model for evaluating third party
trust by both degree of trust and manner of violation. I then applied
this model to a variety of popular third party services as well as
examples of historic trust violations. This analysis is useful in
helping to understand the manners in which user privacy relies on
trusted third parties as well as the motivations that might undercut
this trust. From this analysis, it is clear that user security and
privacy has been, and continues to be, at risk from a wide range of
third part trust violations. Addressing and minimizing this risk is
essential in order to preserve security and privacy of user data in
the digital age.

Toward this end, I provided a number of suggestions for reducing both
the degree of third party trust (e.g. via the use of distributed trust
architectures) as well as for disincentivizing common classes of trust
violations (e.g. by holding third parties liable for such
violations). By taking the multi-pronged approach of both reducing the
degree of power granted to third parties while also creating
disincentives to abusing this power, it is possible to significantly
decrease the degree of risk users expose themselves to while utilizing
third party services. While these techniques are unlikely to fully
eliminate the privacy and security risks inherent to the use of
trusted third parties, they do provide a basis on which such risks can
begin to be measured and mitigated.
