\section{Conclusion}
\label{sec:conclusion}

The pervasiveness of third parities across the modern cloud computing
landscape is undeniable. What this pervasiveness means for the privacy
and security of users and their data is an area of active research. In
this paper, I presented a biaxial model for evaluating third party
trust by both degree of trust and manner of violation. I then applied
this model to a variety of popular third party services as well as
examples of historic trust violations. This analysis is useful in
helping to understand the manners in which user privacy relies on
trusted third parties as well as the motivations that might undercut
this trust. Finally, I provided a number of suggestions for reducing
both the degree of third party trust (e.g. via the use of distributed
trust architectures) as well as for disincentivizing common classes of
trust violations. While these techniques are unlikely to eliminate the
privacy and security risks inherent to the use of trusted third
parties, I hope they provide a basis on which such risks can begin to
be reduced.
