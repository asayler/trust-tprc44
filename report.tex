% Trust
% TPRC 2016 Submission
%
% Spring 2016
%
% Andy Sayler

\documentclass[11pt,letterpaper]{article}

% System Packages
\usepackage{epsfig}
\usepackage{float}
\usepackage[dvipsnames]{xcolor}
\usepackage{caption}
\usepackage{subcaption}
\usepackage{tabu}
\usepackage{hyperref}
\usepackage{url}

% Local Packages
% None

% Package Options
\hypersetup{
    colorlinks,
    citecolor=black,
    filecolor=black,
    linkcolor=black,
    urlcolor=black
}

% Macros
\input{macros.tex}

% Other Options
\clubpenalty = 10000
\widowpenalty = 10000

% Start
\begin{document}

\title{Categorizing, Analyzing, and Mitigating Failures of Third Party Trust}

\author{BLINDED FOR REVIEW}

\date{}

\maketitle

\begin{abstract}

From system administers to service providers to device manufactures,
whom computer users must trust and the capabilities with which they
must trust them lies at the heart of many computer security
discussions. The rise of the ``cloud'' as the preferred platform for
most modern computing applications makes questions of trust even more
critical and pressing. The modern computing landscape requires user to
trust a variety of third parties to complete even the most basic of
digital operations. Unfortunately, such trust often turns out to be
misplaced, leading to data leaks, questionable surveillance practices,
and a wide range of related privacy-harming events.

It is desirable from a public policy standpoint to understand the
nature of modern third party trust and to pursue policies that help
individuals control such trust and that minimize the likelihood of
such trust being violated. Toward these ends, this paper presents a
model for analyzing third party trust and the likelihood of trust
violations. It applies this model to the analysis of a variety of well
publicized data breaches that have occurred over the previous ten
years and uses it to categorize the common manners in which third
parties violate the trust user place in them. Finally, this paper
presents a number of proposed techniques, both technological and
policy-based, for helping to mitigate the degree of trust users must
place in many third parties as well as the likelihood that these
parties will violate this trust.

\end{abstract}

\section{Introduction}
\label{sec:intro}

Over the last decade, computing has undergone a monumental shift from
storing and processing data on individually owned personal computers
to storing and processing data on cloud services owned by a multitude
of third parties. This shift has generated many benefits: sharing data
with other users is trivial, multi-modal communication between users
is easy, and computing devices are largely ephemeral and easily
replaced without any significant loss of user data. This transition,
however, has a significant side effect: user data is now stored in a
manner where it is easily accessible to third parties beyond the
user's immediate control. The shift from locally controlled data to
third party controlled data raises a number of questions, especially
with respect to whom users must trust in order to leverage modern
computing services. Can users maintain the privacy of their digital
data without having to trust anyone?  One can imagine scenarios that
maintain privacy without trust, but such scenarios generally involve
only storing data on self-designed, built, and programmed devices that
never leave one's possession. Such an arrangement is, at best,
impractical for the vast majority of users, and at worst, simply not
adequate to satisfy the demands of today. The range of manufacturers,
developers, and service providers inherent in the modern computing
landscape require that users make decisions regarding whom to trust at
every step of any digital interaction in which they partake.

The popularity of the cloud model leads one to believe that most users
are willing to trade the privacy and control afforded by traditional
computing models for the convenience and features cloud-based services
provide. Individuals regularly place their trust in third parties such
as Facebook, Dropbox, Google, and countless others to securely store
their files, relay their communications, or process their data. But is
this trust desirable and well placed? A 2014 Pew Research study found
that over 90\% of American adults feel that they have lost control
over the data they store in the cloud; 80\% are concerned about how
cloud companies are using their data; and 70\% are concerned about the
manner is which the government might access their
data~\cite{pew-privsec14}. Furthermore, the myriad of recently
publicized data leaks at large companies
(e.g.~\cite{apple-icloudleak}) as well as ongoing government
intrusions into third party user data stores
(e.g.~\cite{greenwald-prism}) has propelled the debate over user
privacy to new levels.

These facts raise a number of important questions. With which
capabilities are users required to trust third parties? In what
manners can this trust be violated? Is this trust an implicit
necessity, or are there ways to reduce such trust? And finally, are
there mechanisms that can reduce the likelihood of third parties
violating a user's trust? This paper aims to address these questions
in three parts. Section~\ref{sec:model} presents a model for
qualifying both \emph{degree} of third party trust as well as
\emph{mechanisms} by which that trust can be violated.
Section~\ref{sec:analysis} provides an analysis of the capabilities
users must entrust to third parties to use a variety of cloud
services, as well examples of common classes of trust
violations. Finally, Section~\ref{sec:mitigation} suggests mechanisms
to allow users to reduce the degree by which they must trust
individual third parties, as well as mechanisms for disincentivizing
violations of this trust, all without limiting a user's ability to
leverage modern computing services.

\section{Modeling Trust}
\label{sec:model}

Researchers from a variety of disciplines have proposed a range of
trust definitions and models~\cite{camp2003, flowerday2006,
  grandison2000, sabater2005}. These models range from technical
models for calculating reputations via machine-learning algorithms to
sociological models for exploring legal and societal notions of
trust. In this section, I propose a trust model for exploring the
manner in which users interact with third parties across the modern
computing landscape. In particular, this model aims to provide a basis
for describing how users trust third parties with access to their
digital data and the manners in which this trust might be violated.

Before defining a model for trust, it is useful to define some of the
relevant terms used in this model. To start, I'll define
\textit{trust} as the expectation that a given entity will behave in a
promised manner. \textit{Violations} of trust thus occur whenever said
entity deviates from this expectation. Trust is closely related to two
other properties inherent in modern computing ecosystems: security and
privacy. Like trust, these terms have wide-ranging meanings across a
variety of disciplines. For the purposes of this discussion, I'll
define \textit{security} as the notion of user control over the
behavior of a given system. A \textit{secure system} is thus a system
that behaves in the manner the user desires. Facets of this notion of
security include \textit{confidentiality}, the ability to control who
can read user data, and \textit{authenticity}, the ability to control
who can modify user data. Finally, confidentiality and authenticity
provide a definition of \textit{privacy} as the ability to control
both the access and modification of user data as well as the ability
to control the meta-record of such access or modification.

When users leverage modern computing devices and services, they must
trust third party manufacturers and service providers to be good
stewards of digital data ranging from stored files to location
information to communication messages. The nature of this trust has
two main factors:

\begin{packed_desc}
\item[Degree:] How much trust must a user place in a third party
  (e.g., what capabilities do they allow a third party to exercise
  with respect to user data)?
\item[Violation:] In what manners can the third party violate this
  trust (e.g., how can the third party abuse the capabilities they
  have been granted or why might they be inclined to do so)?
\end{packed_desc}

The security and privacy of a user's data is generally dependent on
these two axes: the higher the degree of trust a user places in a
third party, the more power that party has to subvert the privacy or
security of a user's data. Similarly, the higher the risk of third
party trust violations, the higher the risk of adverse effects to
security or privacy.  Intuitively, the best ways to enhance the
security and privacy of user data is thus to minimize degree of third
party trust, to minimize the likelihood of third party trust
violations, or to minimize both.

\subsection{Degree of Trust}

Degrees of trust measure the capabilities a third party can exert over
user data. I propose that third parties can be trusted with the
following data-related capabilities:

\begin{packed_desc}
\item[Storage (S-Capability):] \hfill \\ Can a third party faithfully
  store user data and make it available to the user upon request?
  Misuse of this capability may result in a loss of user data, but
  won't necessarily result in the exposure of user data.
\item[Access (R-Capability):] \hfill \\ Can a third party read and
  interpret the user data they store? Misuse of this capability may
  result in the unapproved exposure of user data.
\item[Manipulation (W-Capability):] \hfill \\ Can a third party modify
  the private user data to which they have access? Misuse of this
  capability may result in the ability to manipulate a user
  (e.g., changing appointments on a user's calendar, etc).
\item[Meta-analysis (M-Capability):] \hfill \\ Can a third party
  gather metadata related to any user data or a user's behavior
  interacting with this data? Misuse of this capability may result in
  the ability to infer information about a user (e.g., a user's
  friends).
\end{packed_desc}

While there are likely additional capabilities users can entrust to
third parties, this collection represents the core set of data-related
capabilities most commonly entrusted to cloud service providers.

\subsection{Trust Violations}

Trust violation occurs when a third party exercises any of the above
capabilities without explicit user knowledge and consent. Put another
way, a trust violation occurs whenever a third party leverages a
capability with which they are entrusted in a manner in which the user
does not expect the capability to be leveraged. I propose classifying
such violations into four high-level categories. Each category is
defined by the manner in which the violation occurs and the
motivations behind it:

\begin{packed_desc}
\item[Implicit (P-Violation):] \hfill \\ This class of trust violation
  occurs when a third party violates a user's trust in a manner
  approved by the third party. An example might be sharing user data
  with a business partner (e.g. an advertiser). Often these violations
  aren't really ``violations'' since a user may have clicked through a
  Terms of Service agreement that ``granted'' permission for such use,
  but if the third party is willfully engaging in behavior that the
  user would not generally expect, an implicit trust violation has
  occurred.
\item[Compelled (C-Violation):] \hfill \\ This class of trust
  violation occurs when a third party is compelled by another actor to
  violate a user's trust. The most common example would be a third
  party being forced to turn over user data or records in response to
  a request from the government with jurisdiction over the party.
\item[Unintentional (U-Violation):] \hfill \\ This form of violation
  occurs when a third party unintentionally discloses or manipulates
  user data. An example would be a coding error that allows unfettered
  access to user data. Traditional ``hacking'' attacks also fall into
  this class insofar that such attacks are often possible due to
  unintentional flaws in the design of a ``secure'' system.
\item[Colluding (L-Violation):] \hfill \\ This class of violation
  occurs when multiple third parties collude to gain capabilities over
  user data beyond what the user intended each to have
  individually. An example of such a violation might occur if a user
  has granted two separate parties access to different portions of
  user data (e.g., location data stored with their cellular service
  provider and credit card transaction data stored with their bank)
  that could be combined to reveal more about the user than the user
  intended either party to know.
\end{packed_desc}

While this list of violation categories is far from exhaustive, it
does provide a good high-level framework for exploring the patterns
underlying trust violations and potential methods of mitigation.

\section{Analysis of Services}
\label{sec:analysis}

The trust model proposed in \S~\ref{sec:model} is primarily useful for
what it can tell us about the nature of user trust and the modern
computing landscape. As such, I apply the model to analyze the
capabilities granted to a number of popular third party computing
services.

\subsection{Third Party Capabilities}
\label{sec:analysis:capabilites}

Third party based ``cloud'' computing services have become extremely
popular over the previous 10 years.  The question of how
\textit{trustworthy} these services are is addressed later in
~\S~\ref{sec:analysis:violations}. In this section, I explore how
\textit{trusted} such service are. That is, how much trust must users
place in such services? The capability axis of my proposed model is
useful to quantify this trust.

\subsubsection{Cloud File Storage}

Cloud file storage is a popular third party use case. Services such as
Dropbox~\cite{dropbox}, Google Drive~\cite{google-drive}, and
Microsoft OneDrive~\cite{microsoft-onedrive} all provide users with
mechanisms for storing their files in the cloud, often for the purpose
of keeping files synced across multiple devices or to provide the
ability to share files or collaboratively edit them with other
users. Traditional cloud storage services such as Dropbox, Drive, and
OneDrive are similar enough in their operation that I will use Dropbox
as a stand in for the analysis of all three.

What capabilities is a normal Dropbox user entrusting to Dropbox?
Clearly, users must trust Dropbox to faithfully store their data since
that is Dropbox's core purpose. Users therefore grant Dropbox the
\emph{S} capability. Furthermore, users must also grant Dropbox the
ability to read and access their data (i.e. the \emph{R} capability)
in order to support Dropbox's sharing and syncing features. While
Dropbox doesn't generally utilize it, users are also effectively
granting Dropbox the manipulation (\emph{W}) capability as well since
the user has no mechanisms for ensuring that Dropbox can't manipulate
their data. Finally, Dropbox has full access to user metadata related
to their usage of the service, granting them the \emph{M}
capability. Therefore, Dropbox users must trust Dropbox with all
possible capabilities. Traditional cloud storage services such as
Dropbox, Drive, and OneDrive are thus classified as ``fully trusted''
services: service that require the highest possible level of user
trust. Such services, are thus also in a position to do the greatest
degree of damage to user privacy should a users trust in them as
faithful stewards of private data turn out to be misplaced.

The level of trust requested by tradition cloud storage servers
rightfully makes some user nervous or unwilling to use such
services. In response to such aversion, a number of systems have been
developed with the aim of overcoming third party trust challenges in
the storage space. These systems include ``end-to-end'' encrypted file
storage services such as Tresorit~\cite{tresorit}, or
SpiderOak~\cite{spideroak}. These systems aim to place limits on a
third party's ability to leverage the access (\emph{R}) capability
through the use of client-side encryption. Likewise, they aim to limit
third party access to the manipulation (\emph{W}) capability through
the use of client-side cryptographic
authentication.\footnote{E.g. asymmetric cryptographic signatures such
  as those provided by GnuPG~\cite{gnupg} or symmetric cryptographic
  message authentication codes (MACs) available via a variety of
  algorithms~\cite{dworkin2005, dworkin2007, dworkin2008}.}  In the
base case where a user merely wishes to store data on a single device
and not share it with others, these systems are fairly successful in
achieving their desired trust mitigations. In order to sync data
across multiple devices using such systems, a user must manually
provide some secret (e.g. a password, etc) on each device to secure
its operation. While potentially burdensome and inconvenient, this
practice is in line with these services trusted capabilities
mitigation since it does not require any additional third party trust.

The place where these systems falter at mitigating third party trust
is via their support for multi-user sharing and collaboration. Such
services tend to accomplish multi-user sharing by acting as a trusted
certificate authority (CA) in charge of issuing user
certificates.\footnote{A certificate is a combination of a user's
  public key and certain metadata signed by a trusted issuer. See for
  more information.}These certificates are then used with various
asymmetric cryptographic primitives) to exchange the necessary secrets
for bootstrapping sharing between users. Unfortunately, as a trusted
CA, these services are capable of issuing fraudulent user certificates
to themselves or other parties. This allows them to mount
man-in-the-middle (MitM) attacks on any user trying to share data by
impersonating the recipient of the shared data. This deficiency is
discussed in depth at~\cite{wilson2014}, and leads to a breakdown of
such services' claim that their users need not trust them, at least
when employing multi-user sharing. By mounting a MitM attack on a user
trying to share data with another user, such service providers can
regain the \emph{R} and \emph{W} capabilities they claim not to
have. Furthermore, these services do little to mitigate their access
to metadata (\emph{M} capability). Nor do they provide ways for users
to avoid data loss in the event that one of the services goes offline
or shuts down (\emph{S} capability).

``Secure'' cloud file storage service such as Tresorit this do more to
minimize the required degree of third party trust then traditional
services such as Dropbox. In single-user scenarios, such services
succeed at reducing the degree of user trust from full (all four
capabilities) to partial (only requiring the \emph{S} and \emph{M}
capabilities). Yet, when implementing multi-user use cases, such
services fall back to requiring a more-or-less full degree of trust,
leaving much to be desired.

%%  LocalWords:  OneDrive FISC Tresorit SpiderOak MACs MitM

\section{Managing Trust}
\label{sec:mitigation}

The current trust situation inherent in using many cloud --
i.e. trusting many third parties with a wide range of capabilities and
only moderate disincentivizes to violating user trust -- is far from
ideal. This state places private user data and metadata at a high
degree of risk for unapproved exposure or manipulation. It is natural
to ask what solutions might aid in better controlling third party
trust arraignments, reducing the degree of risk involved when
leveraging third party services. While there are a myriad of potential
solutions in this space, ranging from technical to policy-based, I
suggest a few high level approaches to managing third party trust and
minimizing third party trust violations in this section.

The trust model presented in \S~\ref{sec:model} discusses two axis of
third party trust: the capabilities we entrust to third parties and
the manners in which this trust might be violated. Both axis can be
targeted when seeking to increase the security and privacy of user
data. By reducing the degree or trust -- i.e. limiting the number of
capabilities third parties are granted -- users can limit the amount
of harm a third party can inflict should this trust be violated. By
disincentivizing the various types of trust violations, a user can
decreased the likelihood that a third party violates their trust at
all. I'll discuss on strategies for pursuing both of these goals
below.

\subsection{Limiting Capabilities}
\label{sec:mitigation:capabilites}

Limiting the number of capabilities granted to third parties is an
obvious way to reduce the risk of privacy harms due to third party
trust violations. Furthermore, controlling which capabilities to
entrust to a third party is largely within the control of individual
end users, making this a relatively direct manner in which to reduce
the risk of harm from third trust violations. In the most extreme
case, users can simply elect to avoid using third party services,
effectively granting third parties no data-related capabilities at
all. For most users, however, such an approach is at best impractical,
and in some cases, simply not possible. Therein lies the crux of third
party capability reduction -- simply reducing capabilities is not
enough. Instead, users must have a way to both reduce capabilities
while also maintaining the ability to benefit from third party
services in the manners to which theory are accustomed. Thus, the true
aim of third party trust reduction is to identify ``trust surpluses''
-- situations where third parties are being trusted with more
capabilities than are strictly necessary to provide the benefits the
user derives from the service. Finding and eliminating such surpluses
allows users to reduce the degree by which they must trust third
parties while also continuing to leverage third party services to
provide desirable benefits.

Fortunately (at least from the perspective of users hoping to find
ways to reduce the amount they must trust third parties), trust
surpluses appear to be relatively common in modern third party
services. Take, for example, the Dropbox file syncing service. As
discussed in \S~\ref{sec:analysis:capabilites}, users must currently
entrust Dropbox with all available capabilities: storage (\emph{S}),
access (\emph{R}), modification (\emph{W}), and metadata
(\emph{M}). In order to provide Dropbox's core service, however,
Dropbox only requires a single capability: storage. Thus, granting
Dropbox the access, modification, and metadata capabilities represent
a trust surplus that can conceivably be eliminated without reducing
Dropbox's ability to provide the syncing and sharing benefits users
expect.\footnote{The techniques discussed herein focus primarily on
  limiting surplus access and manipulation
  capabilities. Unfortunately, limiting the metadata capability is
  historically much more difficult than limiting capabilities such as
  access or manipulation. Thus, until better solutions present
  themselves, it may be necessary to continue granting third party
  service providers the metadata capability -- even in surplus.}

The question then becomes how best to limit Dropbox's access to these
surplus capabilities. As mentioned previously, client-side
cryptographic techniques provide tools for liming the access
capability (e.g. encryption) as well as the modification capability
(e.g. authentication). In the case of Dropbox, a client could encrypt
and authenticate their data prior to uploading it to Dropbox and then
decrypt and verify the data when retrieving it from Dropbox. Dropbox
is unable to read or modify such encrypted and authenticated data when
stored on their servers. Such techniques, however, have a
downside. Namely, they require the user to mange and maintain certain
secrets to which Dropbox is not privy -- namely the private keys
necessary to perform data encryption or authentication. Furthermore,
the user must find a way to manually distribute these keys across any
device from which they wish to access their Dropbox files, or share
them with any user with which they wish to share their Dropbox
files. These requirements impose an additional burden on the user,
violating the original premise that users should be able to reduce
third party trust without also reducing their ability to derive
benefits from third party services. Such burdens significantly reduce
the ease of use that draws so many users to solutions such as Dropbox.

There are mechanisms, however, that allow users to both leverage
cryptographic techniques to limit Dropbox's capabilities while also
avoiding the need to impose additional usability burdens on users. For
example, the user could turn to an additional third party Secret
Storage as a Service (SSaaS) service capable of automatically storing,
syncing, and sharing user secrets such as cryptographic keys on the
user behalf. Such a service is discussed in depth
at~\cite{custos-trios}. When used in conjunction with a traditional
cloud storage provider such as Dropbox and existing cryptographic
techniques, a secret storage service can be employed to transparently
limit third party trust without imposing any additional burden on the
end user~\cite{sayler-phd}. In such an arrangement, the end user
stores only encrypted and authenticated file data with Dropbox,
limiting Dropbox's access to the \emph{R} and \emph{W}
capabilities. The user than stores the associated cryptographic
secrets with a secret store provider (SSP) capable of controlling
access to the secrets in a user defined manner and syncing or sharing
them as requested. Neither Dropbox (called a ``feature Provider'' (FP)
in the SSaaS model due to the fact that they primary exist to provide
an end-user with a feature-focused service) nor the SSP have the
ability to access or manipulate user data since Dropbox lacks the keys
necessary to perform such operations and the SSP lacks the data on
which these operations are to be performed.
Figure~\ref{fig:mitigation:trust} illustrates such an
arrangement. Thus, the user has successfully eliminated two of the
surplus capabilities traditionally granted to Dropbox in a manner that
allows them to continue using Dropbox to sync and share files as they
are accustom.

\begin{figure}[t]
  \centering
  \begin{subfigure}[t]{0.48\textwidth}
    \centering
    \includegraphics[height=2in]{./figs/out/TrustModel-Traditional.pdf}
    \caption{Traditional Trust Relationship}
    \label{fig:mitigation:trust:traditional}
  \end{subfigure}
  ~
  \begin{subfigure}[t]{0.48\textwidth}
    \centering
    \includegraphics[height=2in]{./figs/out/TrustModel-Seperated.pdf}
    \caption{Distributed Trust Relationship}
    \label{fig:mitigation:trust:distributed}
  \end{subfigure}
  \caption{Trust Relationships}
  \label{fig:mitigation:trust}
\end{figure}

Techniques such as these are a from of ``trust distribution'' --
e.g. a technique for reducing trust in individual third parties by
instead spreading it across multiple parties. Similar techniques have
been used within cryptographic protocols for the purpose of
eliminating single-points-of-trust~\cite{shamir1979}.\footnote{Such
  techniques also bear some resemble to previously proposed ``key
  escrow systems'', albeit with a somewhat opposite
  end-goal~\cite{denning1996}: escrow systems aim to allow additional
  third parties access to user data whereas trust distribution systems
  aim to reduce the access to user data any single party can achieve.}
Trust distribution techniques are capable of allowing users to reduce
or eliminate trust surpluses across a range of use cases without
introducing significant additional usage burdens. While there are
approaches to limiting third party trust that aim to avoid trusting
any third party (e.g. the OTR chat protocol~\cite{otr-v3}), such
techniques are often difficult to apply generally or to use without
imposing additional usability burdens on end users. Trust distribution
strategies, however, provide a relatively generic framework for
eliminating trust in any single third party.\footnote{When coupled
  with techniques such as~\cite{shamir1979}, trust reduction
  techniques can eliminate trust in even larger subsets of all
  involved parties, e.g. not having to trust up to three of any five
  parties.}

To summarize, the proposed recipe of reducing the number of trusted
capabilities afforded to third parties is as follows:

\begin{packed_enum}
\item Identity any surplus trusted capabilities
\item Leverage cryptographic techniques to limit third party access to
  these capabilities
\item Leverage trust distribution techniques such as SSaaS to store
  and control access to any secret materials required by the
  aforementioned cryptographic techniques in a manner that avoids
  burdening the end user with the need to manage such secrets
  manually.
\end{packed_enum}

This process eliminates trust surplusess by distributing user trust
across multiple third parties in a manner such that individual third
parties can not subvert this trust. As mentioned in
\S~\ref{sec:analysis:violations}, such arrangements do have the
potential to encourage collusion-type trust violations where multiple
third parties work together to regain capabilities that have bean
denied to them. Mechanisms for disincentivizing such violations will
be discussed in the next section.

\subsection{Disincentivizing Violations}
\label{sec:mitigation:violations}

Beyond limiting the number of capabilities users must entrust to third
parties, it is also desirable to disincentive the mechanism by which
third parties might violate such trust.  While technological solution
provide options for reducing degree of trust, it is largely policy
solution that will drive the disincentivization of common classes of
trust violations. By disincentivizing certain classes of trust
violations, we can reduce the likelihood that third parties will
commit such violations, leading to more ``trustworthy'' third parties
and fewer instances of trust violations. There are a variety of
mechanism that one might employ with an aim toward disincentivizing
trust violations. I discusses several of the more prominent ones in
this section.

\subsubsection{Distributed Trust Markets}

In today's single-trust relationships, users primarily select third
party services on the basis of their features. When users pay for
these services, they're primarily paying to support the core features
such services provide. Privacy and security, while potential concerns,
are at best secondary goals. Furthermore, on many free cloud services,
the ability to harvest user data is the basis of the service
provider's business model. As discussed in
Section~\ref{sec:mitigation:capabilites}, these situations create a
number of perverse incentives in terms of a third party's respect to
user security and privacy. In the first case, the third party simply
does not prioritize user security since that is not the primary basis
on which users are choosing to pay for a service. In the second case,
a third party service provider actively works to subvert user security
and privacy in order to further leverage user data to generate income.

Distributed trust relationships such as those provided by the SSaaS
architecture aim to rectify these issues by introducing additional
third party actors whose primary goal is the protection of user
secrets and from whom users purchase secret storage services on the
basis of security and privacy guarantees. The ability of distributed
trust architectures to separate secret storage duties from feature
provider duties allows users to purchase each service on the basis of
its associated merits, avoiding the issues associated with putting
features in direct competition with security and privacy -- a
competition that security and privacy have historically lost. Thus,
distributed trust relationships not only allow users to eliminate
trust surpluses, they also allow users to escape from the artificial
trade off between desirable feature sets and the control and privacy
of their data. Given such separation, independent markets can form
around feature provision and secret protection, optimized for the
respective priorities of each field.

Beyond the removal of perverse incentives brought about by the
distribution of trust across multiple parties, it is also desirable to
encourage a competitive market amongst multiple secret storage
providers. In order to achieve such a market, it is necessary to
standardize a single multi-party-compatible distributed trust
protocol. Such a standard protocol gives users a high degree of
mobility between competing secret storage providers, avoiding vendor
lock-in. This mobility, in turn, increases the competitive pressures
between providers. In short, the aim of a distributed trust ecosystem
is to make security and privacy tradable commodities, and to leverage
market powers to price and improve both. A competitive market for
secret storage has a number of security and privacy enhancing
benefits:

\begin{packed_desc}
\item[Reputation:] If users can easily switch between secret storage
  providers, it forces such providers to compete on the basis of their
  security and privacy preserving reputations. Providers who can do a
  superior job avoiding the trust violations discussed in
  \S~\ref{sec:analysis:violations} can attract more users and/or
  command a higher price for their services. Since a secret storage
  provider reputation is tied solely to their ability to faithfully
  protect user secrets, they will not be able to ``iron over'' any
  privacy-related reputation failings with superior end user feature
  sets -- a practice employed by many traditional feature
  providers.\footnote{As an example, consider Facebook's numerous
    trust violations~\cite{goel2014, lomas2014, tsukayama2014} and the
    fact that such violations have had no noticeable impact on the
    number of people using Facebook~\cite{foster2014}. An secret
    storage provider would enjoy no such network benefit from
    providing additional services beyond secret storage were they to
    violate user's trust; instead, users would simply switch to a new
    provider.}
\item[Multiple Providers:] A healthy ecosystem of competing secret
  storage providers will allow users to select from multiple
  independent providers over which they may further distribute their
  trust beyond a binary feature provider + secret storage provider
  relationship. Such sharding provides a number of benefits over
  relying on a single SSP, from additional trust reduction to data
  redundancy.
\item[Cost:] As in other competitive markets, having a number of
  competing providers will allow the user to select a provider that
  offers the best combination of cost and service.
\end{packed_desc}

Distributed trust markets are potentially useful at disincentivizing a range
of trust violations, from implicit violations to unintentional
violations. Such markets help to align economic incentives that
practices that disfavor such violations.

\subsubsection{Digital Due Process}

While mechanisms such as distributes trust markets are useful for
disincentivizing many classes of trust violations, other mechanisms
are needed to disincentive compelled violations. Trust markets
potentially encourage third parties to push back against compelled
trust violations to the maxim extent permitted under the law, but they
do little to protect users in cases where the law heavily favors
compelled trust violations even in cases where such violations are not
in the public interest. To reduce such violations, it is necessary to
protect ``digital due process'' rights.

The first step toward protecting such rights is to ensure that user
data stored or processed by third party services receives the same
level of protection as data stored or processed locally. This concept
runs counter to the Third Party Doctrine established bu current
U.S. case law~\cite{thompson-thirdparty}. This doctrine holds that
individuals who voluntarily store their data with third parties have
no ``reasonable expectation of privacy''~\cite{scotus-katzvus} for
such data. While such a viewpoint may have made sense in the mid
20\textsuperscript{th} century when it was established via a series of
Supreme Court rulings~\cite{scotus-usvmiller-privacy,
  scotus-smithvmaryland}, it does not translate well to a world where
third party access to user data is the norm. As shown in
\S~\ref{sec:analysis:violations} compelled violations are a growing
trend, and in many cases, such violations are served via third-party
doctrine mechanisms. Such trends suggest a likely overreach of
government data collection, leading to a range of potential adverse
effect (e.g.~\cite{penney2016}). One possible way of halting or
reversing this trend would be to eliminate the third party doctrine
and begin requiring a 4\textsuperscript{th} Amendment warrant in order
to compel third to provide or modify user data.

Fortunately, changes to the third party doctrine are beginning to
progress on multiple fronts. Recent Supreme Court decisions have
suggested a willingness to expand user privacy rights in the digital
realm, and may eventually lead the Supreme Court to revisit the third
party doctrine~\cite{scotus-usvjones}.  Congress has also long been
departing updates to third-party doctrine derived laws such a the
Electronic Communications Privacy Act (ECPA)~\cite{ecpa} to include a
warrant requirement for digitally stored
emails~\cite{eff-ecpareform}. Recently, the U.S. House of Represents
even unanimously passed a bill to amend the ECPA to required warrants
in most cases~\cite{trujillo-ecpa}. Such movements suggest a growing
recognition of the due process rights of digital data, regardless of
whether it is stored locally or by third parties. Such trends likely
represent the best hope for reducing unnecessary compelled violations,
ensuring such violations only occur in cases where the public interest
in significantly favored by the violation of user trust.

\subsubsection{Third Party Liability}

Another mechanism for disincentivizing trust violations, especially of
the unintentional variety, would be to establish standards of
liability for trust violations resulting in misuse of the capabilities
with which they have been entrusted. If a third party violates a user
trust, it is reasonable to expect that users should be able to seek
some measure of relief for such violations. Such liability would
follow the growing trend toward holding companies liable for digital
data breaches resulting from poor security practices
(e.g.~\cite{ftc-asus}).

The nature of this liability could take several forms. The most
obvious form would be to impose civil liability commensurate with the
value of of the harm caused by a trust violation on the party
committing said violation. This opens up the thorny issue of how to
value such harms. For example, a trust violation resulting in the
public exposure of a piece of user data may result in fairly minimal
harm (e.g. the leaking of a set of not particularly sensitive photos
taken in a public space), or it could result in fairly serious harms
(e.g. the leaking of trade secret or other sensitive
material)~\cite{acquisti2013, romanosky2009}. One way to overcome the
harm valuation challenge would be to have users declare the value of
the harm that would result from the misuse of trust capabilities when
entrusting third parties with such capabilities. This approach is
similar to the manner in which one might declare the value of a parcel
when shipping it for the purpose of securing insurance. Given such a
declaration, the third parties could charge a user varying amounts:
entrusting third parties with access to more ``valuable'' data should
increase the cost of the service to the end user, while entrusting
access to less ``valuable'' data should cost less. Damages in the
event that a secret is lost could then be calculated as a multiple of
this value. In cases where a trust violation occurs due to an
unforeseeable event or otherwise through no negligence on the part of
the third party (e.g. a flaw in a respected third party library or
similar unintentional trust violation), the user would be reimbursed
the declared value of the harm (or potentially some fraction of
it). In the case where a trust is violated due to third party
negligence, malpractice, or malfeasance, (e.g. an implicit or
particularly egregious unintentional trust violation) the user would
be reimbursed several multiples of the declared harm (i.e. statutory
damages).

In addition to allowing users to seek compensation for harm suffered
due to third party trust violations, this approach also favors the use
of distributed trust architecture. Since such architecture reduce the
number of capabilities with which any single third party must be
trusted, they also reduce the declared value of any associated
harms. Assuming that third parties charge users for their services on
the basis of the declared value of the data the user wishes to store,
distributed architecture thus end up reducing the cost of third party
services to the end user by reducing associated harm cost. E.g., by
distributing trust across multiple parties, the user devalues the harm
each party can inflict, allowing the user to declare lower harm costs
and pay less for the third party services they use. This arrangement
both incentivize users to spread their trust across multiple parties
and encourage third parties to avoid any trust violation that would
require them to pay out the associated harm cost.

To manage this liability, third parties would likely be required to
secure insurance to cover the cost of liability payouts in the event
that a trust violation occurs~\cite{ciab2015, starks2016}. These
insurers are in a position to provide additional disincentive to third
party trust violations. Insurers could charge each third party on the
basis of both the harm violation value of the capabilities with which
they have been entrusted, as well as on the basis of how ``secure''
(or the inverse: how ``risky'') a third parties security
capability-protection are. Indeed, it is even possible that the
government itself might act as such an insurer (or underwriter), as
they currently do with banks via the Federal Deposit Insurance
Corporation (FDIC)~\cite{fdic}.

Regardless of mechanism, establishing a standard system for trust
violation liability will help to disincentivize trust violations
across a range of third parties. Fostering policies encouraging the
development of a robust liability insurance market will further
incentivize best practices. Such practices also encourage the use of
distribute trust architecture and the additional trust-preserving
benefits they provide.


%%  LocalWords:  SSaaS SSP OTR FP th ECPA

\section{Conclusion}
\label{sec:conclusion}

The pervasiveness of third parties across the modern cloud computing
landscape is undeniable. What this pervasiveness means for the privacy
and security of users and their data is an area of active research. In
this paper, I presented a bi-axial model for evaluating third party
trust by both degree of trust and manner of violation. I then applied
this model to a variety of popular third party services as well as
examples of historic trust violations. This analysis is useful in
helping to understand the manners in which user privacy relies on
trusted third parties as well as the motivations that might undercut
this trust. From this analysis, it is clear that user security and
privacy has been, and continues to be, at risk from a wide range of
third part trust violations. Addressing and minimizing this risk is
essential in order to preserve security and privacy of user data in
the digital age.

Toward this end, I provided a number of suggestions for reducing both
the degree of third party trust (e.g. via the use of distributed trust
architectures) as well as for disincentivizing common classes of trust
violations (e.g. by holding third parties liable for such
violations). By taking the multi-pronged approach of both reducing the
degree of power granted to third parties while also creating
disincentives to abusing this power, it is possible to significantly
decrease the degree of risk users expose themselves to while utilizing
third party services. While these techniques are unlikely to fully
eliminate the privacy and security risks inherent to the use of
trusted third parties, they do provide a basis on which such risks can
begin to be measured and mitigated.


\bibliographystyle{acm}
\bibliography{refs}

\end{document}
