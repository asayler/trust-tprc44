\section{Analysis of Services}
\label{sec:analysis}

The trust model proposed in \S~\ref{sec:model} is primarily useful for
what it can tell us about the nature of user trust and the modern
computing landscape. As such, I apply the model to analyze the
capabilities granted to a number of popular third party computing
services.

\subsection{Third Party Capabilities}
\label{sec:analysis:capabilites}

Third party based ``cloud'' computing services have become extremely
popular over the previous 10 years.  The question of how
\textit{trustworthy} these services are is addressed later in
~\S~\ref{sec:analysis:violations}. In this section, I explore how
\textit{trusted} such service are. That is, how much trust must users
place in such services? The capability axis of my proposed model is
useful to quantify this trust.

\subsubsection{Cloud File Storage}

Cloud file storage is a popular third party use case. Services such as
Dropbox~\cite{dropbox}, Google Drive~\cite{google-drive}, and
Microsoft OneDrive~\cite{microsoft-onedrive} all provide users with
mechanisms for storing their files in the cloud, often for the purpose
of keeping files synced across multiple devices or to provide the
ability to share files or collaboratively edit them with other
users. Traditional cloud storage services such as Dropbox, Drive, and
OneDrive are similar enough in their operation that I will use Dropbox
as a stand in for the analysis of all three.

What capabilities is a normal Dropbox user entrusting to Dropbox?
Clearly, users must trust Dropbox to faithfully store their data since
that is Dropbox's core purpose. Users therefore grant Dropbox the
\emph{S} capability. Furthermore, users must also grant Dropbox the
ability to read and access their data (i.e. the \emph{R} capability)
in order to support Dropbox's sharing and syncing features. While
Dropbox doesn't generally utilize it, users are also effectively
granting Dropbox the manipulation (\emph{W}) capability as well since
the user has no mechanisms for ensuring that Dropbox can't manipulate
their data. Finally, Dropbox has full access to user metadata related
to their usage of the service, granting them the \emph{M}
capability. Therefore, Dropbox users must trust Dropbox with all
possible capabilities. Traditional cloud storage services such as
Dropbox, Drive, and OneDrive are thus classified as ``fully trusted''
services: service that require the highest possible level of user
trust. Such services, are thus also in a position to do the greatest
degree of damage to user privacy should a users trust in them as
faithful stewards of private data turn out to be misplaced.

The level of trust requested by tradition cloud storage servers
rightfully makes some user nervous or unwilling to use such
services. In response to such aversion, a number of systems have been
developed with the aim of overcoming third party trust challenges in
the storage space. These systems include ``end-to-end'' encrypted file
storage services such as Tresorit~\cite{tresorit}, or
SpiderOak~\cite{spideroak}. These systems aim to place limits on a
third party's ability to leverage the access (\emph{R}) capability
through the use of client-side encryption. Likewise, they aim to limit
third party access to the manipulation (\emph{W}) capability through
the use of client-side cryptographic
authentication.\footnote{E.g. asymmetric cryptographic signatures such
  as those provided by GnuPG~\cite{gnupg} or symmetric cryptographic
  message authentication codes (MACs) available via a variety of
  algorithms~\cite{dworkin2005, dworkin2007, dworkin2008}.}  In the
base case where a user merely wishes to store data on a single device
and not share it with others, these systems are fairly successful in
achieving their desired trust mitigations. In order to sync data
across multiple devices using such systems, a user must manually
provide some secret (e.g. a password, etc) on each device to secure
its operation. While potentially burdensome and inconvenient, this
practice is in line with these services trusted capabilities
mitigation since it does not require any additional third party trust.

The place where these systems falter at mitigating third party trust
is via their support for multi-user sharing and collaboration. Such
services tend to accomplish multi-user sharing by acting as a trusted
certificate authority (CA) in charge of issuing user
certificates.\footnote{A certificate is a combination of a user's
  public key and certain metadata signed by a trusted issuer. See for
  more information.}These certificates are then used with various
asymmetric cryptographic primitives) to exchange the necessary secrets
for bootstrapping sharing between users. Unfortunately, as a trusted
CA, these services are capable of issuing fraudulent user certificates
to themselves or other parties. This allows them to mount
man-in-the-middle (MitM) attacks on any user trying to share data by
impersonating the recipient of the shared data. This deficiency is
discussed in depth at~\cite{wilson2014}, and leads to a breakdown of
such services' claim that their users need not trust them, at least
when employing multi-user sharing. By mounting a MitM attack on a user
trying to share data with another user, such service providers can
regain the \emph{R} and \emph{W} capabilities they claim not to
have. Furthermore, these services do little to mitigate their access
to metadata (\emph{M} capability). Nor do they provide ways for users
to avoid data loss in the event that one of the services goes offline
or shuts down (\emph{S} capability).

``Secure'' cloud file storage service such as Tresorit this do more to
minimize the required degree of third party trust then traditional
services such as Dropbox. In single-user scenarios, such services
succeed at reducing the degree of user trust from full (all four
capabilities) to partial (only requiring the \emph{S} and \emph{M}
capabilities). Yet, when implementing multi-user use cases, such
services fall back to requiring a more-or-less full degree of trust,
leaving much to be desired.

%%  LocalWords:  OneDrive FISC Tresorit SpiderOak MACs MitM
