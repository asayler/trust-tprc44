\section{Introduction}
\label{sec:intro}

Over the last decade, computing has undergone a monumental shift from
storing and processing data on individually owned personal computers
to storing and processing data on cloud services owned by a multitude
of third parties. This shift has generated many benefits: sharing data
with other users is trivial, multi-modal communication between users
is easy, and computing devices are largely ephemeral and easily
replaced without any significant loss of user data. This transition,
however, has a significant side effect: user data is now stored in a
manner where it is easily accessible to third parties beyond the
user's immediate control. The shift from locally controlled data to
third party controlled data raises a number of questions, especially
with respect to whom users must trust in order to leverage modern
computing services. Can users maintain the privacy of their digital
data without having to trust anyone?  One can imagine scenarios that
maintain privacy without trust, but such scenarios generally involve
only storing data on self-designed, built, and programmed devices that
never leave one's possession. Such an arrangement is, at best,
impractical for the vast majority of users, and at worst, simply not
adequate to satisfy the demands of today. The range of manufacturers,
developers, and service providers inherent in the modern computing
landscape require that users make decisions regarding whom to trust at
every step of any digital interaction in which they partake.

The popularity of the cloud model leads one to believe that most users
are willing to trade the privacy and control afforded by traditional
computing models for the convenience and features cloud-based services
provide. Individuals regularly place their trust in third parties such
as Facebook, Dropbox, Google, and countless others to securely store
their files, relay their communications, or process their data. But is
this trust desirable and well placed? A 2014 Pew Research study found
that over 90\% of American adults feel that they have lost control
over the data they store in the cloud; 80\% are concerned about how
cloud companies are using their data; and 70\% are concerned about the
manner is which the government might access their
data~\cite{pew-privsec14}. Furthermore, the myriad of recently
publicized data leaks at large companies
(e.g.~\cite{apple-icloudleak}) as well as ongoing government
intrusions into third party user data stores
(e.g.~\cite{greenwald-prism}) has propelled the debate over user
privacy to new levels.

These facts raise a number of important questions. With which
capabilities are users required to trust third parties? In what
manners can this trust be violated? Is this trust an implicit
necessity, or are there ways to reduce such trust? And finally, are
there mechanisms that can reduce the likelihood of third parties
violating a user's trust? This paper aims to address these questions
in three parts. Section~\ref{sec:model} presents a model for
qualifying both \emph{degree} of third party trust as well as
\emph{mechanisms} by which that trust can be violated.
Section~\ref{sec:analysis} provides an analysis of the capabilities
users must entrust to third parties to use a variety of cloud
services, as well examples of common classes of trust
violations. Finally, Section~\ref{sec:mitigation} suggests mechanisms
to allow users to reduce the degree by which they must trust
individual third parties, as well as mechanisms for disincentivizing
violations of this trust, all without limiting a user's ability to
leverage modern computing services.
