\section{Introduction}
\label{sec:intro}

Trust and privacy are closely related qualities in computing. Can
users maintain their digital privacy without having to trust anyone?
One can imagine scenarios that maintain privacy without trust, but
such scenarios generally involve only storing data on self-designed,
built, and programmed devices that never leave one's possession. Such
an arrangement is, at best, impractical for the vast majority of
users, and at worst, simply not possible to achieve today. The range
of manufactures, developers, and service providers inherent in the
modern computing landscape require that users make decisions regarding
whom to trust at every step of any digital interaction in which they
partake.

The cloud computing model, by its very nature, further amplifies the
number of parties users must trust. Individuals regularly place their
trust in third parties such as Facebook, Dropbox, Google, and
countless others to securely store their files, relay their
communications, or process their data. But is this trust well placed?
A range of recent data breaches effecting entities ranging from health
insurance providers to the U.S. Government can be traced back to
violations of the trust that users place in third parties. This fact
raises a number of additional questions. With which capabilities are
users required to trust third parties? In what manners can this trust
be violated? Is this trust an implicit necessity, or are there ways to
reduce the trust users must place in third parties. And finally, are
there policy mechanisms that can be used to reduce the likelihood of
third party trust violations.

This paper aims to discuss and provide answers to these questions.

\subsection{Background}

Trust has long been a key component of computing
security~\cite{thompson1984}.
